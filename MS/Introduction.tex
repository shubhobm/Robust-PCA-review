\section{Introduction}

Principal component Analysis (PCA) is one of the oldest, yet most widely used methods of unsupervised multivariate analysis. For a data matrix $\bfX \in \BR^{n \times p}$ containing observations in $p$ variables for $n$ samples, each column having mean 0, principal components are defined as $p$-dimensional vectors $\bfw_k, 1 \leq k \leq p$ such that
%
\begin{align}\label{eqn:eqnPCA}
\bfw_k &=
\begin{cases}
\argmax_{\| \bfw \| = 1} \bfw^T \bfX^T \bfX \bfw \text{ if }k = 1\\
\argmax_{\| \bfw \| = 1} \bfw^T \bfR_k^T \bfR_k \bfw \text{ if }k > 1;
\quad \bfR_k = \bfX - \sum_{s=1}^{k-1} \bfX \bfw_s \bfw_s^T
\end{cases} 
\end{align}
%
Following a lagrange multiplier approach, the eigenvectors of $\bfX^T \bfX$, equivalently the right singular vectors obtained from the singular value decomposition of $\bfX$ provide solutions to (\ref{eqn:eqnPCA}).