%\documentclass[fleqn,11pt]{article}
%%\usepackage{mycommands1,color,caption,hyperref,amsmath,amssymb,amsthm,mathabx}
%%\usepackage[pdftex]{graphicx}
%%\usepackage[round]{natbib}
%%\usepackage{subfigure,epsfig}
%
%\usepackage{mycommands1,amssymb,amsmath,amsthm,color,pagesize,outlines,cite,subfigure,epsfig}
%\usepackage[small]{caption}
%\usepackage{hyperref} % for linking references 
%\usepackage{stackrel}
%
%\usepackage[round]{natbib}
%
%% for algorithm
%\usepackage[noend]{algpseudocode}
%\usepackage{algorithm}
%
%%% Appendix theorem counter
%\usepackage{chngcntr}
%\usepackage{apptools}
%\AtAppendix{\counterwithin{Theorem}{section}}
%
%\numberwithin{equation}{section}
%
%% measurements for 1 inch margin
%\addtolength{\oddsidemargin}{-.875in}
%\addtolength{\evensidemargin}{-.875in}
%\addtolength{\textwidth}{1.75in}
%\addtolength{\topmargin}{-.875in}
%\addtolength{\textheight}{1.75in}
%
%\usepackage{hyperref} % for linking references 
%\usepackage{setspace}
%\doublespacing

% Modified for use with Wiley Interdisciplinary Reviews Copyright (C) (2016). All rights reserved.
\documentclass[12pt]{article}

\setlength{\oddsidemargin}{0in}  %left margin position, reference is one inch
\setlength{\textwidth}{6.5in}    %width of text=8.5-1in-1in for margin
\setlength{\topmargin}{-0.5in}    %reference is at 1.5in, -.5in gives a start of about 1in from top
\setlength{\textheight}{9in}     %length of text=11in-1in-1in (top and bot. marg.) 

\usepackage{amsmath,amssymb,amsthm,mycommands1}
\usepackage[round]{natbib}
\usepackage{graphicx}% Include figure files
\usepackage{caption}
\usepackage{color}% Include colors for document elements
\usepackage{dcolumn}% Align table columns on decimal point
\usepackage{bm}% bold math
\usepackage{float}
\usepackage{hyperref} % hyperref must be loaded before apacite
%\usepackage{apacite}
%\bibliographystyle{apacite}
\hypersetup{ colorlinks = true, citecolor = blue, urlcolor = blue}
%\usepackage[nolists, nomarkers, figuresfirst]{endfloat}

%% For pstricks
\usepackage[pdf]{pstricks}
\usepackage{epsfig}
\usepackage{pst-grad} % For gradients
\usepackage{pst-plot} % For axes
\usepackage[space]{grffile} % For spaces in paths
\usepackage{etoolbox} % For spaces in paths
\makeatletter % For spaces in paths
\patchcmd\Gread@eps{\@inputcheck#1 }{\@inputcheck"#1"\relax}{}{}
\makeatother

\definecolor{background-color}{gray}{0.98}

\title{Robust Principal Component Analysis: a Review}
% The title should not exceed 20 words. Please be original and try to include keywords, especially before a colon if applicable, as they will increase the discoverability of your article. Visit http://media.wiley.com/assets/7158/18/SEO_For_Authors.pdf for tips on search engine optimization.

\author{Subhabrata Majumdar\thanks{Department of Biology, University 1, ...}, Snigdhansu Chatterjee\thanks{Department of Chemistry, University 2, ...}}
% The preferred (but optional) format for author names is First Name, Middle Initial, Last Name.
% Wiley requires that all authors disclose any potential conflicts of interest.  Any interest or relationship, financial or otherwise, that might be perceived as influencing an author’s objectivity is considered a potential conflict of interest. The existence of a conflict of interest does not preclude publication.

%\date{}

\DeclareMathOperator*{\ve}{vec}
\DeclareMathOperator*{\diag}{diag }
\DeclareMathOperator*{\Tr}{Tr}
\DeclareMathOperator*{\argmin}{arg\,min}
\DeclareMathOperator*{\argmax}{arg\,max}

\begin{document}
\maketitle

\newtheorem{Theorem}{Theorem}[section]
\newtheorem{Lemma}[Theorem]{Lemma}
\newtheorem{Corollary}[Theorem]{Corollary}
\newtheorem{Proposition}[Theorem]{Proposition}
\newtheorem{Conjecture}[Theorem]{Conjecture}
\theoremstyle{definition} \newtheorem{Definition}[Theorem]{Definition}

\date{}
%\author{Subhabrata Majumdar, Snigdhansu Chatterjee}
\maketitle

%{(\colrbf will change)} We introduce a composition of the spatial sign function \citep{LocantoreEtal99} with transformations on functions that are essentially the outlyingness maps of \cite{zuo00}, with a few restrictions for technical convenience. After a brief consideration of its performance in the location problem for elliptical distributions, we define a multivariate rank vector using this. We discuss several aspects of its performance in estimating components of the covariance matrix in the data-generating elliptical distribution: its eigenvectors, eigenvalues and the covariance matrix itself. Several simulation studies and data examples outline the utility of these methods, and we also discuss their implementation in Sufficient Dimension Reduction \citep{AdragniCook09} and functional outlier detection.
\vspace{.5cm}

\begin{center}
\subsubsection*{\small Article Type:}
Advanced Review
%The Article Type denotes the intended level of readership for your article. An Editor may have mentioned a specific Article Type in your invitation letter; if so, please let them know if you think a different Article Type better suits your topic.

\hfill \break
\thanks

\subsubsection*{Abstract}
\begin{flushleft}
The abstract should not exceed 250 words and should be a concise description of the article and its implications. It should include all keywords associated with your article, as keywords increase its discoverability. Please do not include phrases such as ``This article discusses \ldots" or ``Here we review \ldots", references to other articles, or URLs.
\end{flushleft}
\end{center}

\subsubsection*{Keywords}
Principal Component Analysis; Robustness; ROBPCA; Spatial signs; data depth; Principal Component Pursuit

\clearpage

\renewcommand{\baselinestretch}{1.5}
\normalsize

\clearpage

%
\section*{\sffamily \Large INTRODUCTION}

Principal component Analysis (PCA) is one of the oldest, yet most widely used methods of unsupervised multivariate analysis. Given a $p$-dimensional random variable $\BX$ with mean vector ${\bf 0}_p$ and covariance matrix $\bfSigma$, the principal component transformation is defined as:
%
\begin{align}
\BX \mapsto \BY = \bfGamma^T \BX
\end{align}
%
where $\bfGamma$ is orthogonal, and $\bfGamma^T \bfSigma \bfGamma = \bfLambda = \diag( \lambda_1, \ldots, \lambda_p); \lambda_1 \geq \ldots \geq \lambda_p \geq 0$. This induces a set of linear transformations on the random vector $\bfX$ so that the transformations are uncorrelated with each other, and their variances ordered from highest to smallest \citep{MardiaKentBibby}. Given that $\bfSigma$ is positive definite, these transformations are given by the eigenvectors of $\bfSigma$ following its spectral decomposition. For a size-$n$ sample from the distribution of $\BX$, say $\bfX = (\bfx_1, \ldots, \bfx_n)^T$, the first principal component gives the linear combination of columns of $\bfX$ which maximizes sample variance:
%
\begin{align}\label{eqn:eqnPCA1}
\bfw_1 &= \argmax_{\| \bfw \| = 1} Var (\bfX \bfw ) = \argmax_{\| \bfw \| = 1} \bfw^T \bfX^T \bfX \bfw
\end{align}
%
and the subsequent principal components are defined as
\begin{align}\label{eqn:eqnPCAk}
\bfw_k &= \argmax_{\| \bfw \| = 1} \bfw^T \bfR_k^T \bfR_k \bfw; \quad \bfR_k = \bfX - \sum_{s=1}^{k-1} \bfX \bfw_s \bfw_s^T \quad \text{for } 1< k \leq p
\end{align}
%
Following a lagrange multiplier approach, the eigenvectors of $\bfX^T \bfX/n$, equivalently the right singular vectors obtained from the singular value decomposition of $\bfX$ provide solutions to (\ref{eqn:eqnPCA1}) and (\ref{eqn:eqnPCAk}).

PCA has seen extensive applications in diverse areas, such as image recognition \citep{AlkandariAljaber15}, finance \citep{AlexanderBook}, climate science \citep{WilksBook} and text mining \citep{BerryCastellanos}, with the objective being reduction of the intrinsic dimensionality of the feature space without losing information. However, because the objective function to be maximized is quadratic, PCA performs poorly in presence of even a small proportion of corrupted observations \citep{XuCaramanisMannor13}. Based on the domain of application, these corruptions can be the result of data heterogeneity \citep{SahaEtal16}, measurement error \citep{Bailey12,HelltonThoresen14}, or may represent structured noise \citep{CandesEtal09}.

Depending on the modelling goals, robust PCA aims to estimate principal components or the underlying low dimensional subspace in presence of corrupted entries in the data matrix. Historically, the instrumental factors behind the evolution of robust PCA methods have been the size and complexity of datasets, availability of computational resources, as well as the nature of corruptions present in the data. Early methods of robust PCA were focused on robustly estimating the population covariance matrix from datasets that are small to moderate in size, and comprised of independent samples: some of which were outliers, i.e. contained corrupted entries. Later on, the computational and statistical challenges that surfaced with the advent of high-dimensional datasets having a large number of features were tackled by methods like projection pursuit \citep{LiChen85}, ROBPCA \citep{hubert05} and M-estimation \citep{LocantoreEtal99,Majumdar15}.

This article is organized as follows. The theoretical discussion is composed of two sections, and we discuss the above broad approaches of robust PCA on independent data in further detail in the first of those sections. We devote the other section to Principal Component Pursuit (PCP), which, even though introduced very recently \citep{CandesEtal09}, has motivated a substantial amount of research on the problem of recovering an underlying low-rank structure in the data, rather than the principal components \textit{per se}, in presence of noise. We also illustrate the relevance and relative performance of these two types of methods using two real data examples. Following this we review the methods of robust PCA in domains that are not multivariate real, for example reproducing kernel hilbert spaces or the space of square-integrable functions, in the section {\it Robust PCA in Other Spaces}. We finally finish the review with section {\it Conclusion}. {\colrbf word it better}

\section{Robust covariance estimation, data transformation, and beyond}
\label{Section:sec2}

\subsection{Robust covariance matrices, projection pursuit}

The earliest approaches to robust PCA were based on robustly estimating the population covariance matrix, and using eigenvectors of that estimate as principal components. Some methods of robust covariance matrix estimation include the Minimum Volume Ellipsoid estimator \citep{Rousseeuw84}, a projection-based estimator by \citep{maronna76}, the Minimum Covariance Determinant (MCD) estimator \citep{rousseeuw85} and the Stael-Donoho estimators \citep{MaronnaYohai95,ZuoCui05}. Although these estimators have high breakdown points, they suffered from two severe drawbacks. Firstly the explicit evaluation of the population covariance matrix meant that obtaining principal components were not possible when $n < p$. Secondly, even when $n > p$, these methods become computationally intensive with large data dimensions.

\subsection{Data transformation}

\subsection{Robust PCA and outlier detection}
Aside from obtaining a lower dimensional projection of the data matrix $\bfX$ in spite of outliers that is close enough to the projection of $\bfX$ by the first few population eigenvectors, detecting the outliers themselves is also closely associated with robust PCA. These samples can be of interest for mechanistic reasons. For example in the analysis of near infra-red absorbance for 39 gasoline samples over 226 wavelengths using ROBPCA \citep{hubert05}, six compounds are flagged as outliers, and these turn out to be the only samples containing alcohol. \cite{hubert05} also introduced a notion of outlier diagnostics that is applicable to any method of robust PCA and can serve as a means to compare different relevant techniques as well.

We illustrate this in \ref{fig:figROBPCA}. Here we consider data in 3 dimensions, and consider the relative position of the samples with respect to the two-dimensional principal component subspace $\cM$. We can classify such points into four categories:

\begin{enumerate}
\item{\it Regular observations}: points that form a homogeneous group close to $\cM$ ($A$ and $B$ in figure);
\item{\it Good leverage points}: points that lie close to $\cM$, but at a distance from the regular observations ($C$ in figure);
\item{\it Orthogonal outliers}: These points (point $D$ in figure) lie far away from their projections on $\cM$ (point $D'$, but the projections themselves are close to the regular observations;
\item{\it Bad leverage points}: These points are also far away from their projections on $\cM$ ($E$ and $E'$ respectively), but the projections are also far away from the regular observations.
\end{enumerate}

\cite{hubert05} introduced the concept of \textit{score distance} (SD) and \textit{orthogonal distance} (OD) to distinguish between these four types of points. With our notation, for the $i^\text{th}$ observation these distances are defined as:
%
$$
SD_i = \sum_{j=1}^q \frac{t_{ij}}{\lambda_j};\
\quad OD_i  =\| ( \bfI - \bfW_k \bfW_k^T) (\bfx_i - \bfmu) \|
$$
%
The SD can be interpreted as the weighted distance of the projection of a point on the hyperplane formed by the first $k$ PCs, while OD is the orthogonal distance of that point and the $k$-PC hyperplane. It is now clear from our picture that regular observations have low values of both SD and OD, while bad leverage points have high values of both. An orthogonal outlier has small SD but large OD, whereas a good leverage point has high SD but small OD. To explicitly classify sample points into these 4 categories, \cite{hubert05} use $\sqrt {\chi^2_{k,0.975}}$ and $[ \hat \mu (OD^{2/3}) + \hat \sigma (OD^{2/3}) \Phi^{-1} (0.975) ]^{3/2}$ as upper cutoffs for score distance and orthogonal distance, respectively. Here $\hat \mu$ and $\hat \sigma$ are univariate MCD estimators, and $\Phi$ is the standard normal cumulative distribution function.
\section*{\sffamily \Large PRINCIPAL COMPONENT PURSUIT}
\label{section:sec3}

The above notion of outliers depends on the fact that the $n \times p$ data matrix $\bfX$ is composed of observations from several independent samples in its rows, and some of these samples have corrupted observations. However, in many practical situations, rows of $\bfX$ might not be independent, the corrupted observations can have a pattern across samples, or both. For example in face or handwriting recognition, each individual picture can be taken as a data matrix. The value of a pixel takes corresponds to an entry in the data matrix, with noisy pixels denoting corrupted measurements. Although the underlying low-rank structure is still of interest in such situations, for example the face of a person or a handwritten digit, this problem is fundamentally different because of the inherent structure present in the data.

\cite{CandesEtal09} first introduced {\it Principal Component Pursuit} (PCP), which decomposes the data matrix into low-rank and sparse components to tackle the above situation. Formally, PCP considers the following additive model:
%
\begin{align}\label{eqn:PCPmodel}
\bfX = \bfL_0 + \bfS_0
\end{align}
%
with rank$(\bfL_0) = r < p$ and $\bfS_0$ sparse. The low-rank and sparse structures are recovered using nuclear norm penalization on the first component and $\ell_1$-norm penalization on the second component, respectively:
%
\begin{align}\label{eqn:PCPobj}
& \text{minimize } \| \bfL \|_* + \| \bfS \|_1; \quad \text{subject to } \bfL + \bfS = \bfX
\end{align}
%
where $\|.\|_*$ denotes the nuclear norm of a matrix, i.e. sum of its singular values, and $\|.\|_1$ denotes $\ell_1$-norm, i.e. sum of the absolute values of its entries. \cite{CandesEtal09} proved that given the true underlying structure is indeed low-rank-plus-sparse, i.e. adheres to the decomposition in (\ref{eqn:PCPmodel}), a polynomial time algorithm based on convex programming can exactly recover these matrices, and this is possible for arbitrary magnitudes of entries in the sparse component.

\subsection*{\sffamily \large PCP and matrix completion}
Both the polynomial time algorithm and arbitrary magnitude of corrupted entries are strengths of PCP over traditional methods of robust PCA. Another reason the PCP is attractive by itself is because with slight modifications, it can perform robust matrix completion. The matrix completion problem attempts to fill in a data matrix $\bfY$ through nuclear norm minimization when only a subset $\Omega \subset \{ 1, \ldots, n\} \times \{ 1, \ldots, p\}$ of its entries are observed. Formally stated, the problem amounts to
%
$$
\text{minimize } \| \bfL \|_*; \quad \text{ subject to } \bfP_\Omega \bfL = \bfY
$$
%
where $\bfP_\Omega$ is the known indicator matrix of non-missing entries: $(\bfP_\Omega)_{ij} = \BI_{(i,j) \in \Omega}$. PCP simply assumes there is a sparse noise component in the incomplete data: $\bfY = \bfP_\Omega (\bfL_0 + \bfS_0)$, and recovers the low-rank structure:
%
\begin{align}\label{eqn:PCPMCobj}
& \text{minimize } \| \bfL \|_* + \| \bfS \|_1; \quad \text{subject to } \bfP_\Omega(\bfL + \bfS) = \bfY
\end{align}
%

\cite{CandesEtal09} showed that it is possible to solve this problem with minimal modifications to their original PCP algorithm that solves (\ref{eqn:PCPobj}). Multiple further studies provided improvements for several aspects of this basic setup. The work of \cite{ChenEtal11} is prominent among them. In particular, they assumed the presence of both errors and missing entries, with deterministic or random support for each of them, and provided theoretical performance guarantees when the fraction of observed entries vanishes as $n \rightarrow \infty$. They also performed worst-case analysis for the errors-only or missing-only scenarios.

\subsection*{\sffamily \large Modifications}
In a subsequent paper, \cite{ZhouEtal10} added an entrywise noise component $\bfZ$ to the objective functions in (\ref{eqn:PCPobj}):
%
\begin{align}\label{eqn:PCPobjZ}
& \text{minimize } \| \bfL \|_* + \| \bfS \|_1; \quad \text{subject to } \bfL + \bfZ + \bfS = \bfX
\end{align}
%
and (\ref{eqn:PCPMCobj}):
%
\begin{align}\label{eqn:PCPMCobjZ}
& \text{minimize } \| \bfL \|_* + \| \bfS \|_1; \quad \text{subject to } \bfP_\Omega(\bfL + \bfZ + \bfS) = \bfY
\end{align}
%
This brought the PCP formulation closer to the classical robust PCA setup that separates a lower-dimensional component in presence of both data-wide additional noise and corrupted entries, with the advantage that here the magnitude and structure of corrupted entries can be arbitrary. Further modifications of PCP include the case when the lower-dimensional component is a union of multiple lower dimensional subspaces \citep{WohlbergEtal12}, adding an $\ell_1/\ell_2$-penalization term on $\bfL$ \citep{TangNehorai11}, a dual formulation of the problem \citep{BeckerEtal11}, and non-convex robust matrix completion \citep{ShangEtal14}. PCP has been an active area of research in the signal and image processing community for the past few years. Please refer to \cite{Bouwmans14} for a detailed review on more modifications of the PCP, algorithmic developments, and its applications in video surveillance.
\section{Numerical examples}
\label{section:sec4}
\section{Robust PCA in other spaces}
\label{section:Others}

\subsection{Kernel PCA}

\subsection{Functional PCA}
\section*{\sffamily \Large CONCLUSION}

In the above sections we have reviewed several methods available in the literature concerned with recovery of an underlying low rank structure in the data matrix in presence of atypical data points. The data examples illustrate that robust PCA is not a `one-method-fits-all' problem, and care should be exercised on what techinque should be applied on what data. Most of these literature is devoted towards robustness in presence of outlying samples. We believe robustness towards other factors, like model misspecification and missing data, needs to be further explored. In functional data, many of the methods proposed are robust against outlying curves that do not conform to the shape of the other curves, but not many that can detect outlying points in an otherwise typical curve. As mentioned in \cite{BaliBoenteReview}, this is a challenging problem and needs more attention.


\bibliographystyle{apalike}
\bibliography{reviewbib}

\end{document}